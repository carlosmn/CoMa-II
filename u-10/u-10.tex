\documentclass[ngerman,a4paper]{scrartcl}
\usepackage{relsize}
\usepackage{fullpage}
\usepackage[german]{babel}
\usepackage{graphicx}
\usepackage{cancel}

%Compiler
\usepackage{ifxetex}
\usepackage{ifluatex}
\ifxetex
  \usepackage{fontspec,xunicode}
  \catcode`\ß=13
  \defß{\ss}
\else\ifluatex
  \usepackage{fontspec,xunicode}
\else
  \usepackage[utf8]{inputenc}
\fi\fi
% /Compiler

\usepackage{amsmath, amssymb, amsfonts, amsthm}
\usepackage{listings}
\lstset{frame=single}

\newcommand{\norm}[1]{\left|\!\left|#1\right|\! \right|}
\newcommand{\R}{\ensuremath{\mathbb{R}}}
\newcommand{\N}{\ensuremath{\mathbb{N}}}

\begin{document}
{\sffamily
  \hfill
  CoMa-II SS 2013\hfill
  FU Berlin\\[8pt]
  \noindent {\Huge Übung 10}\hfill Carlos Martín Nieto, Tran Tu\hrule \bigskip
}

\section*{1}

\paragraph{b)}

Wir zeigen durch einfaches einsetzen. Zuerst berechnen wir $\dot{y}(t)$ "`direkt".

\[
  y(t) =
  \begin{pmatrix}
    \sin(ct)\\ \cos(ct)
  \end{pmatrix}
  \qquad
  \dot{y}(t) =
  \begin{pmatrix}
    \cos(ct)\cdot c\\ -\sin(ct) \cdot c
  \end{pmatrix}
\]

und dann durch $Ay(t)$.
\[
  Ay(t) =
  \begin{pmatrix}
    0 & c\\ -c & 0
  \end{pmatrix}
  \begin{pmatrix}
    \sin(ct)\\ \cos(ct)
  \end{pmatrix} =
  \begin{pmatrix}
    c \cos(ct)\cdot c\\ c -\sin(ct)
  \end{pmatrix}
\]

und analog für $\tilde{y}(t)$. Zuerst direkt
\[
  \tilde{y}(t) =
  \begin{pmatrix}
    -\cos(ct)\\ \sin(ct)
  \end{pmatrix}
  \qquad
  \dot{\tilde{y}}(t) =
  \begin{pmatrix}
    \sin(ct) \cdot c\\ \cos(ct)\cdot c
  \end{pmatrix}
\]

und dann durch $A\tilde{y}(t)$.
\[
  A\tilde{y}(t) =
  \begin{pmatrix}
    0 & c\\ -c & 0
  \end{pmatrix}
  \begin{pmatrix}
    -\cos(ct)\\ \sin(ct)
  \end{pmatrix} =
  \begin{pmatrix}
    c \sin(ct) \\ c \cos(ct)
  \end{pmatrix}
\]

\paragraph{c)}

\begin{align*}
  y(0) =
  \begin{pmatrix}
    \alpha \\ \beta
  \end{pmatrix} &= C_1
  \begin{pmatrix}
    \sin(0)\\ \cos(0)
  \end{pmatrix} + C_2
  \begin{pmatrix}
    -\cos(0)\\ \sin(0)
  \end{pmatrix}\\
  &= C_1
  \begin{pmatrix}
    0 \\ 1
  \end{pmatrix} + C_2
  \begin{pmatrix}
    -1\\ 0
  \end{pmatrix}\\
\alpha = -C_2 \implies C_2 &= -\alpha\\
\beta &= C_1\\
\implies y(t) &= \beta
\begin{pmatrix}
  \sin(ct)\\ \cos(ct)
\end{pmatrix} - \alpha
\begin{pmatrix}
  -\cos(ct) \\ \sin(0)
\end{pmatrix}
\end{align*}
\section*{2}

Zur Erinnerung, die Trapezregel ist


\begin{align*}
  x(t+\tau) &= x(t) + \frac{\tau}{2} (f(t) + f(t + \tau))\\
  &= x(t) + \frac{\tau}{2} (Ax(t) + Ax(t + \tau))\\
  \intertext{Wir müssen wieder nach $x(t+\tau)$ lösen}\\
  x(t+\tau) &= x(t) + \frac{\tau}{2} Ax(t) + \frac{\tau}{2} Ax(t + \tau)\\
  x(t+\tau) - \frac{\tau}{2} Ax(t + \tau)&= x(t) + \frac{\tau}{2} Ax(t)\\
  \left[E  -\frac{\tau}{2} A\right] x(t+\tau) &= \left[E + \frac{\tau}{2} A\right] x(t)\\
  \left[E  -\frac{\tau}{2} A\right]^{-1} \left[E  -\frac{\tau}{2} A\right] x(t+\tau) &= \left[E  -\frac{\tau}{2} A\right]^{-1} \left[E + \frac{\tau}{2} A\right] x(t)\\
  x(t+\tau) &= \left[E  -\frac{\tau}{2} A\right]^{-1} \left[E + \frac{\tau}{2} A\right] x(t)
\end{align*}

\section*{3}

Fixpunktiteration löst Gleichungen der Form

\[
f(x) = 0
\]

Wir haben aber

\[
g(x) = y
\]

Dies lässt sich leicht lösen, indem wir $f$ so definieren

\[
f(x) = g(x) - y
\]

dass $=0$ gilt. Da $y$ eine Konstante ist, ist die Ableitung von
$f(x)$ auch die Ableitung von $g(x)$.

\end{document}
