\documentclass[a4paper,ngerman]{scrartcl}
\usepackage{listings}
\usepackage[ngerman]{babel}
\usepackage{dsfont}
\usepackage{amsmath}
\usepackage{amssymb}
\usepackage{wasysym}
\usepackage{cancel}
\usepackage{graphicx}

\usepackage{tikz}
\usepackage{tikz-qtree}

\usepackage{fullpage}

%Compiler
\usepackage{ifxetex}
\usepackage{ifluatex}
\ifxetex
  \usepackage{fontspec,xunicode}
  \catcode`\ß=13
  \defß{\ss}
\else\ifluatex
  \usepackage{fontspec,xunicode}
\else
  \usepackage[utf8]{inputenc}
\fi\fi
% /Compiler

\newcommand{\R}{\ensuremath{\mathds{R}}}%
\newcommand{\N}{\ensuremath{\mathds{N}}}%
\newcommand{\kabs}{\ensuremath{\kappa_{\text{abs}}}}%
\newcommand{\krel}{\ensuremath{\kappa_{\text{rel}}}}%
\newcommand{\krels}[1]{\ensuremath{\kappa_{\text{rel}_#1}}}%
\newcommand{\BO}{\ensuremath{\mathcal{O}}}%

\begin{document}
\title{CoMa-II}
\subtitle{1. Zettel}
\author{Carlos Martín Nieto \and Tran Tu}
\maketitle

\section*{Aufgabe 1}
\paragraph{a)}
Aus dem gegebenem können wir folgendes Gleichungssystem erstellen

\begin{align*}
  a_3 \left(\frac{-1}{2}\right)^3 + a_2\left(\frac{-1}{2}\right)^2 + a_1\frac{-1}{2} + a_0 &= \frac{3}{4}\\
  a_3 \cdot 0 + a_2 \cdot 0 + a_1 \cdot 0 + a_0 &= 1\\
  a_3 \cdot 1 + a_2 \cdot 1 + a_1 \cdot 1 + a_0 &= 0\\
  a_3 \cdot 2^3+ a_2 \cdot 2^2+ a_1 \cdot 2+ a_0 &= -3
\end{align*}

was als Matrizen so aussieht:

\[
\underbrace{\begin{pmatrix}
  \left(\frac{-1}{2}\right)^3 & \left(\frac{-1}{2}\right)^2 & \frac{-1}{2} & 1\\
  0 & 0 & 0 & 0\\
  1 & 1 & 1 & 1\\
  2^3 & 2^2 & 2 & 1
\end{pmatrix}}_{A}\cdot
\underbrace{\begin{pmatrix}
  a_3\\ a_2\\ a_1\\ a_0
\end{pmatrix}}_{X} =
\underbrace{\begin{pmatrix}
  \frac{3}{4} \\ 1 \\ 0 \\ -3
\end{pmatrix}}_{B}
\]

Matlab löst das für uns

\begin{lstlisting}
  octave:1> aufgabe1
ans =

   0
  -1
  -0
   1
\end{lstlisting}

und so wissen wir, dass unser interpolierter Polynom

\[
p(x) = -x^2 + 1
\]

ist.

\paragraph{b)}
Die Ordnung des Polynoms ist kleiner als die Anzahl der Stützstellen.


\section*{Aufgabe 2}

\paragraph{a)}
\[
  L_1(x_j) =
  \begin{cases}
    1 & x_1\\
    0 & x_2\\
    0 & x_3
  \end{cases} \qquad
  L_2(x_j) =
  \begin{cases}
    0 & x_1\\
    1 & x_2\\
    0 & x_3
  \end{cases} \qquad
  L_3(x_j) =
  \begin{cases}
    0 & x_1\\
    0 & x_2\\
    1 & x_3
  \end{cases}
\]

Das Polynom in Lagrange-Darstellung sieht dann so aus

\begin{align*}
  p_n(x) = \sum^n_{k=0} \alpha_k\cdot L_k(x) = L_1(x) -3\cdot L_3(x)
\end{align*}

Wir berechnen $L_1$ und $L_3$ für $x=\frac{-1}{2}$

\begin{align*}
  L_1(x) = \prod^n_{\substack{j = 0\\j \neq k}} \frac{x - x_j}{x_i -
    x_j} &= \frac{x - x_2}{x_1 - x_2} \cdot \frac{x - x_3}{x_1 - x_3}
  = \frac{\frac{-1}{2} - 1}{0 -1}\cdot \frac{\frac{-1}{2} - 2}{0 -2}
  = \frac{\frac{-3}{2}}{-1} \cdot \frac{\frac{-5}{2}}{-2} =
  \frac{3}{2} \cdot \frac{5}{4} = \frac{15}{8}\\
  L_3(x) = \prod^n_{\substack{j = 0\\j \neq k}} \frac{x - x_j}{x_i -
    x_j} &= \frac{x - x_1}{x_3 - x_1} \cdot \frac{x - x_2}{x_3 - x_2}
  = \frac{\frac{-1}{2} - 0}{2 - 0}\cdot \frac{\frac{-1}{2} - 1}{2 - 1}
  = \frac{\frac{-1}{2}}{2} \cdot \frac{\frac{-3}{2}}{1} =
  \frac{-1}{4} \cdot \frac{-3}{2} = \frac{3}{8}
\end{align*}

und somit

\[
p_n\left(\frac{-1}{2}\right) = L_1\left(\frac{-1}{2}\right) - 3 \cdot
L_3\left(\frac{-1}{2}\right) = \frac{15}{8} -
3\left(\frac{3}{8}\right) = \frac{15}{8} - \frac{9}{8} = \frac{6}{8} =
\frac{3}{4}
\]

\paragraph{b)}
Die Newtonische Darstellung hat die Form

\[
p_l(x) = a_0 + \sum^l_{i = 1} a_i \prod^{i-1}_{k = 1} (x-x_k)
\]

und zuerst müssen wir $a_0,\dots,a_l$ mittels dividierende Differenzen
berechnen

\[
\begin{array}{lll}
  \overbrace{f[x_1] = 1}^{a_0} & \overbrace{f[x_1,x_2]}^{a_1} = -1 & \overbrace{f[x_1,x_2,x_3]}^{a_2} = -1\\
  f[x_2] = 0 & f[x_2,x_3] = -3\\
  f[x_3] = -3
\end{array}
\]

Nebenrechnungen:
\begin{align*}
  f[x_1,x_2] &= \frac{f[x_2] - f[x_1]}{x_2 - x_1} = \frac{0 - 1}{1 - 0} = -1\\
  f[x_2,x_3] &= \frac{f[x_3] - f[x_2]}{x_3 - x_2} = \frac{-3 - 0}{2 - 1} = -3\\
  f[x_1,x_2,x_3] &= \frac{f[x_2,x_3] - f[x_1,x_2]}{x_3 - x_1} =
  \frac{-3 - (-1)}{2 - 0} = \frac{-2}{2} = -1
\end{align*}

und somit ist die Darstellung

\begin{align*}
  p_l(x) &= 0 + 1\cdot (x-x_1)(x-x_2)(x-x_3) - 1 \cdot (x-x_1) (x-x_2)\\
  & (x-x_3) -1\cdot (x-x_1)(x-x_2)(x-x_3)\\
  p_l(x) &= x(x-1)(x-2) - x(x-1)(x-2) - x(x-x_2)(x-x_3)
\end{align*}

\section*{Aufgabe 3}

\paragraph{d)}

Hier, \texttt{testfun} ist eine Funktion, die die Werte aus der
2. Aufgabe zurück gibt.

\lstinputlisting{tests.txt}

\end{document}
