\documentclass[a4paper,ngerman]{scrartcl}
\usepackage{listings}
\usepackage[ngerman]{babel}
\usepackage{dsfont}
\usepackage{amsmath, amsthm}
\usepackage{amssymb}
\usepackage{wasysym}
\usepackage{cancel}
\usepackage{graphicx}

\usepackage{tikz}
\usepackage{tikz-qtree}

\usepackage{fullpage}

%Compiler
\usepackage{ifxetex}
\usepackage{ifluatex}
\ifxetex
  \usepackage{fontspec,xunicode}
  \catcode`\ß=13
  \defß{\ss}
\else\ifluatex
  \usepackage{fontspec,xunicode}
\else
  \usepackage[utf8]{inputenc}
\fi\fi
% /Compiler

\newcommand{\R}{\ensuremath{\mathds{R}}}%
\newcommand{\N}{\ensuremath{\mathds{N}}}%
\newcommand{\kabs}{\ensuremath{\kappa_{\text{abs}}}}%
\newcommand{\krel}{\ensuremath{\kappa_{\text{rel}}}}%
\newcommand{\krels}[1]{\ensuremath{\kappa_{\text{rel}_#1}}}%
\newcommand{\BO}{\ensuremath{\mathcal{O}}}%

\begin{document}
\title{CoMa-II}
\subtitle{2. Zettel}
\author{Carlos Martín Nieto \and Tran Tu}
\maketitle

\section*{Aufgabe 2}

Für $n=0$ ist es Trivial, da es nur $x_0$ gibt. Für $n=1$ gibt nur
zwei mögliche Reihenfolgen ($[x_0,x_1]$ und $[x_1,x_0]$). Wir beweisen
es so:

\begin{align*}
  f[x_0,x_1] &= \frac{f[x_1] - f[x_0]}{x_1-x_0} =
  \frac{f[x_1]}{x_1-x_0}-\frac{f[x_0]}{x_1-x_0} = 
  \overbrace{\frac{f[x_0]}{x_0-x_1} + \frac{f[x_1]}{x_1-x_0}}^{(\star)}=\\
  &= \frac{f[x_0]}{x_0-x_1} - \frac{f[x_1]}{x_0-x_1} =\frac{f[x_0] - f[x_1]}{x_0-x_1} = f[x_1,x_0]
\end{align*}

Allgemeiner, $f[x_0,\dots,x_n]$ kann auch so geschrieben werden (mit $i$ der aktueller Index)
\[
f[x_0,\dots,x_n] = \frac{f[x_0]}{\displaystyle\prod^{n}_{\scriptsize{
    \begin{array}{cc}
      j=0\\j\neq i
    \end{array}
}} (x_i-x_j)} + \dots +
\frac{f[x_n]}{\displaystyle\prod^{n}_{\scriptsize{
    \begin{array}{cc}
      j=0\\j\neq i
    \end{array}
}} (x_i-x_j)} =
\sum^n_{i=0} f[x_i]\left[ \prod^n_{\scriptsize{
    \begin{array}{cc}
      j=0\\j\neq i
    \end{array}
}}\frac{1}{(x_i-x_j)}\right]
\]

und daher spielt die Reihenfolge keine Rolle.

\begin{proof}
  Per Induction. $n=1$ wurde schon in $( \star)$ gezeigt. Es gilt zu beweisen, dass wenn
  $n\to n+1$, diese Formel immer noch gültig ist.
\begin{align*}
  f[x_0,\dots,x_{n+1}] &= \frac{f[x_1,\dots,x_{n+1}] - f[x_0,\dots,x_n]}{x_{n+1}-x_0}=\\
  &=\frac{1}{x_{n+1}-x_0}f[x_1,\dots,x_{n+1}]) - \frac{1}{x_{n+1}-x_0}f[x_0,\dots,x_n]\\
  &= \frac{1}{x_{n+1}-x_0}f[x_1,\dots,x_{n+1}]) + \frac{1}{x_{0}-x_{n+1}}f[x_0,\dots,x_n]\\
\end{align*}
Wir wissen schon wie $n+1$-elementige dividierte Differenzen
aussehen. In diesem Fall mit $n=2$
\begin{align*}
f[x_0,x_1,x_2]  &= \frac{1}{x_{2}-x_0}\left[ \frac{f[x_1]}{x_1-x_2} + \frac{f[x_2]}{x_2-x_1}\right] -
\frac{1}{x_{2}-x_0}\left[ \frac{f[x_0]}{x_0-x_1} + \frac{f[x_1]}{x_1-x_0}\right] =\\
&= \frac{f[x_0]}{(x_0-x_1)(x_0-x_2)} + \frac{f[x_1]}{(x_1-x_0)(x_1-x_2)} + \frac{f[x_2]}{(x_2-x_1)(x_2-x_0)}
\end{align*}

Die $f[x_0]$ und $f[x_2]$ Terme sind offensichtlich. Die Berechnung
des $f[x_1]$-Terms ist etwas umständlicher, und zwar
\begin{align*}
&  \frac{1}{x_2-x_0}\left[\frac{f[x_1]}{(x_1-x_2)}-\frac{f[x_1]}{(x_1-x_0)}\right] =
\frac{1}{x_2-x_0}\left[\frac{f[x_1](x_1-x_0)-f[x_1](x_1-x_2)}{(x_1-x_2)(x_1-x_0)}\right] =\\
=& \frac{1}{x_2-x_0}\left[ \frac{f[x_1](\cancel{x_1}-x_0\cancel{-x_1}+x_2)}{(x_1-x_2)(x_1-x_0)} \right] =
\frac{1}{\cancel{x_2-x_0}}\cdot \frac{\cancel{x_2-x_0}}{1}\cdot \frac{f[x_1]}{(x_1-x_0)(x_1-x_2)}
\end{align*}
\end{proof}

\section*{Aufgabe 3}

\paragraph{b)}

Es gibt keine wesentliche Unterschiede zwischen die Genauigkeit bei 10
und bei 100 Stützstellen. Bei 100 Stützstellen gibt es in der Nähe von
$-2$ Ungenauigkeiten (ein oder zwei Werte sind sehr groß), die im
\texttt{HP}-Graph nicht zu sehen sind, da sonst die meisten Werte alle
gleich aussehen würden.

\section*{Aufgabe 4}

Wie bei Aufgabe 3, bilden sich am Rande einige Ungenauigkeiten. Bei
den Stützstellen in der Nähe von 2 (mit 100 Stützstellen) sinken die
Werte statt zu steigen.

\end{document}
