\documentclass[ngerman,a4paper]{scrartcl}
\usepackage{relsize}
\usepackage{fullpage}
\usepackage[german]{babel}
\usepackage{graphicx}
\usepackage{cancel}

%Compiler
\usepackage{ifxetex}
\usepackage{ifluatex}
\ifxetex
  \usepackage{fontspec,xunicode}
  \catcode`\ß=13
  \defß{\ss}
\else\ifluatex
  \usepackage{fontspec,xunicode}
\else
  \usepackage[utf8]{inputenc}
\fi\fi
% /Compiler

\usepackage{amsmath, amssymb, amsfonts, amsthm}
\usepackage{listings}
\lstset{frame=single}

\newcommand{\norm}[1]{\left|\!\left|#1\right|\! \right|}
\newcommand{\R}{\ensuremath{\mathbb{R}}}
\newcommand{\N}{\ensuremath{\mathbb{N}}}

\begin{document}
{\sffamily
  \hfill
  CoMa-II SS 2013\hfill
  FU Berlin\\[8pt]
  \noindent {\Huge Übung 6}\hfill Carlos Martín Nieto, Tran Tu\hrule \bigskip
}

\section*{2}

\paragraph{a)}


Nach der Lösung der letzten Woche haben wir
\begin{align*}
  x(t) &= C_1 \cos(t) + C_2 \sin(t)\\
  x(0) &= \boxed{C_1 = x_0}\\
  x'(t) &= -C_1 \sin(t) + C_2 \cos(t)\\
  x'(t) &= \boxed{C_2 = 0}
\end{align*}

Wir setzen dies in $E(x,x')$ ein:
\begin{align*}
  E(x,x') &= \sqrt{x_0^2 \sin^2(t) + x_0^2 \cos^2(t)}\\
  &= \sqrt{x_0^2 (\sin^2(t) + \cos^2(t))}\\
  &= \sqrt{x_0^2} = x_0\\
  E(x,x') &= 0
\end{align*}

\paragraph{b)}


\begin{align*}
  \begin{array}{c}
    x_1(t) =\\
    x_2(t) =
  \end{array}
  \begin{pmatrix}
    C_1 \cos(t)\\
    C_1 \sin(t)
  \end{pmatrix}
&= C_1 \cdot
\begin{pmatrix}
  \cos(t)\\ 0
\end{pmatrix} + C_1 \cdot
\begin{pmatrix}
  0\\ -\sin
\end{pmatrix}\\
&= x_0 \sin(t)
\begin{pmatrix}
  0\\ -1
\end{pmatrix} + x_0 \cos(t)
\begin{pmatrix}
  1\\ 0
\end{pmatrix}\\
\begin{pmatrix}
  \cancel{x_0} \sin'(t)\\ \cancel{x_0} \cos'(t)
\end{pmatrix} &=
\begin{pmatrix}
  \cancel{x_0} \sin(t)\\ \cancel{x_0} \cos(t)
\end{pmatrix} \cdot
\begin{pmatrix}
  0 & 1\\-1 & 0
\end{pmatrix}
\end{align*}

Die Funktion E würde jetzt so aussehen

\[
E = \text{Skalaprodukt}\{{-x_0 \cdot \sin, x_0 \cdot \cos}\}
\]

\end{document}
