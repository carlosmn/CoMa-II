\documentclass[ngerman,a4paper]{scrartcl}
\usepackage{relsize}
\usepackage{fullpage}
\usepackage[german]{babel}
\usepackage{graphicx}
\usepackage{cancel}

%Compiler
\usepackage{ifxetex}
\usepackage{ifluatex}
\ifxetex
  \usepackage{fontspec,xunicode}
  \catcode`\ß=13
  \defß{\ss}
\else\ifluatex
  \usepackage{fontspec,xunicode}
\else
  \usepackage[utf8]{inputenc}
\fi\fi
% /Compiler

\usepackage{amsmath, amssymb, amsfonts, amsthm}
\usepackage{listings}
\lstset{frame=single}

\newcommand{\norm}[1]{\left|\!\left|#1\right|\! \right|}
\newcommand{\R}{\ensuremath{\mathbb{R}}}
\newcommand{\N}{\ensuremath{\mathbb{N}}}

\begin{document}
{\sffamily
  \hfill
  CoMa-II SS 2013\hfill
  FU Berlin\\[8pt]
  \noindent {\Huge Übung 9}\hfill Carlos Martín Nieto, Tran Tu\hrule \bigskip
}

\section*{2}

Von der letzten Übung wissen wir

\[
x_{k+1} = x_k + \underbrace{\frac{\lambda \tau}{2} (x_k + x_{k+1})}_{\int\limits^{t+\tau}_t \lambda x \, dt}
\]

Daher gilt

\[
\varepsilon(t_k, \tau) = x_{k+1} - x_k - \frac{\lambda\tau}{2} (x_k + x_{k+1}) \quad (1)
\]

Da


\begin{align*}
  x'(t) &= \lambda x(t) \implies \frac{dx}{dt} \lambda x \quad t_k +
  \tau t_{k+1}\\
  &\implies \int^{t_{k+1}}_{t_k} \frac{dx}{dt} dt = \int^{t_{k+1}}_{t_k} \! \lambda x\, dt\\
  &\implies x(t_{k+1}) - x(t_k) = \lambda \int^{t_{k+1}}_{t_k} \! x\, dt \quad (*)\\
  \intertext{Trapezregel}\\
  \lambda \int^{t_{k+1}}_{t_k} \! x\, dt &= \frac{\lambda\tau}{2} [x(t_{k+1}) + x(t_k)]
\end{align*}

Zuzüglich der Fehlerdarstellung der Trapezregel $\frac{\tau^3}{12}x''(\xi)$


\begin{align*}
  (*) \quad x(t_{k+1}) - x(t_k) &= \frac{\lambda\tau}{2}[x(t_{k+1}) +
  x(t_k)] + \frac{\tau^3}{12} x''(\xi)\\
  x(t_k + 1) = x(t_k) + \frac{\lambda\tau}{2} [x(t_{k+1}) +
  x(t_k)] + \frac{\tau^3}{12} x''(\xi)\\
\end{align*}

Wir setzen (2) in (1) ein

\begin{align*}
  \varepsilon(t_k, \tau) &= \cancel{x_k} + \cancel{\frac{\lambda\tau}{2} (x_{k+1} + x_k)}
  + \frac{\tau^3}{12} x''(\xi) - \cancel{x_k} - \cancel{\frac{\lambda\tau}{2} (x_{k} + x_{k+1})}\\
  &= \frac{\tau^3}{12} x''(\xi)
\end{align*}

Daraus folgt

\[
\max_{k=0,\dots,n-1} |\varepsilon(t_k, \tau)| \leq \frac{1}{12} \max_{t \in (0, T]} | x''(t)| \tau^3
\]

$\implies$ $p=2$ mit $C = \frac{1}{12} ||x''||_\infty$

\end{document}
