\documentclass[ngerman,a4paper]{scrartcl}
\usepackage{relsize}
\usepackage{fullpage}
\usepackage[german]{babel}
\usepackage{graphicx}
\usepackage{cancel}

%Compiler
\usepackage{ifxetex}
\usepackage{ifluatex}
\ifxetex
  \usepackage{fontspec,xunicode}
  \catcode`\ß=13
  \defß{\ss}
\else\ifluatex
  \usepackage{fontspec,xunicode}
\else
  \usepackage[utf8]{inputenc}
\fi\fi
% /Compiler

\usepackage{amsmath, amssymb, amsfonts, amsthm}
\usepackage{listings}
\lstset{frame=single}

\newcommand{\norm}[1]{\left|\!\left|#1\right|\! \right|}
\newcommand{\R}{\ensuremath{\mathbb{R}}}
\newcommand{\N}{\ensuremath{\mathbb{N}}}

\begin{document}
{\sffamily
  \hfill
  CoMa-II SS 2013\hfill
  FU Berlin\\[8pt]
  \noindent {\Huge Übung 8}\hfill Carlos Martín Nieto, Tran Tu\hrule \bigskip
}

\section*{1}

\begin{itemize}
\item[a)] Nein, weil es mehr als eine Variable gibt
\item[b)] Ja, gewöhnlich linear von Ordnung 1
\item[c)] Ja, analog zu \textbf{b)}
\item[d)] Nein, sie ist 2. Ordnung
\end{itemize}
\section*{2}

\paragraph{a)}

Benutzt man die Trapezregel, um die Integral zu berechnen, dann sieht so aus
\begin{align*}
  x(t + \tau) &= x(t) + \frac{t+\tau-t}{2} (f(t) + f(t+\tau))\\
  &= x(t) + \frac{\tau}{2} (f(t) + f(t+\tau))
\end{align*}

\paragraph{b)}

Mit dieser Form der Berechnung von $x(t+\tau)$ haben wir das gleiche
Problem wie beim impliziten Verfahren, der Wert, den wir wollen ist
auf beiden Seiten der Gleichung. Wir formen also um.

Wie beim impliziten Verfahren, wählen wir eine Art von Funktion, mit
der wir Arbeiten. In diesem Fall ist die Funktion der Form $f(t) =
\lambda x(t)$. Es ginge analog mit der $f(t) = \lambda x(t) + g(t)$
wir letzte Woche, aber aus Einfachheit Betrachten wir das diesmal
nicht.

\begin{align*}
  x(t + \tau) &= x(t) + \frac{\tau}{2} \left[f(t) + f(t+\tau)\right]\\
  \intertext{Wir ersetzen $f$ durch ihre generische Form}\\
  &= x(t) + \frac{\tau}{2} \left[ \lambda x(t) + \lambda x(t+\tau) \right]\\
  &= x(t) + \frac{\tau}{2} \lambda x(t) + \frac{\tau}{2} \lambda x(t+\tau)\\
  x(t + \tau) - \frac{\tau}{2} \lambda x(t+\tau) &= x(t) + \frac{\tau}{2} \lambda x(t)\\
  x(t + \tau)\cdot \underbrace{\left[1 - \frac{\tau\lambda}{2}\right]}_{\frac{2-\tau\lambda}{2}} &= x(t) + \frac{\tau}{2} \lambda x(t)\\
  x(t + \tau) &= \frac{x(t) + \frac{\tau}{2} \lambda x(t)}{\frac{2-\tau\lambda}{2}}\\
  x(t + \tau) &= \frac{2x(t) + \tau\lambda x(t)}{2 - \tau\lambda}
\end{align*}

Und diese Gleichung ist was wir im Matlab-Programm implementiert
haben.

\end{document}
