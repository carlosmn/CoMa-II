\documentclass[ngerman,a4paper]{scrartcl}
\usepackage{relsize}
\usepackage{fullpage}
\usepackage[german]{babel}
\usepackage{graphicx}

%Compiler
\usepackage{ifxetex}
\usepackage{ifluatex}
\ifxetex
  \usepackage{fontspec,xunicode}
  \catcode`\ß=13
  \defß{\ss}
\else\ifluatex
  \usepackage{fontspec,xunicode}
\else
  \usepackage[utf8]{inputenc}
\fi\fi
% /Compiler

\usepackage{amsmath, amssymb, amsfonts, amsthm}
\usepackage{listings}
\lstset{frame=single}

\newcommand{\norm}[1]{\left|\!\left|#1\right|\! \right|}
\newcommand{\R}{\ensuremath{\mathbb{R}}}
\newcommand{\N}{\ensuremath{\mathbb{N}}}

\begin{document}
{\sffamily
  \hfill
  Computerorientierete Mathematik SS 2013\hfill
  FU Berlin\\[8pt]
  \noindent {\Huge Übung 3}\hfill Carlos Martín Nieto, Tran Tu\hrule \bigskip
}

\section*{1}

\paragraph{a)}

Die Funktionswerte oszillieren, was man mit $[-1,1]$ nicht sieht. Beim
erhöhen der Ordnung der Interpolation kann man sehen, dass die Werte
an den "`Grenzen"' sehr stark schwingen und je mehr Stützstellen wir
wählen, desto stärker tun sie es. Es gibt ein Unterschied zwischen der
Interpolationsmethoden. Die klassische Interpolation bleibt stabiler.

\section*{2}

\paragraph{a)}

Zeigen Sie, dass die Newton-Côtes-Formeln für alle
$n\in\N\setminus\{0\}$ symmetrisch sind.

Zur zeigen ist, dass $\lambda_0 = \lambda_{n-0}$, $\lambda_1 =
\lambda_{n-1}$. Also, dass $\lambda_k = \lambda_{n-k}$. Die Gewichte
werden durch
\[
\lambda_k = \frac{1}{n} \int\limits^n_0\prod^n_{\substack{j=0\\j\neq k}} \frac{s-j}{k-j}\text{d}s
\]
gegeben. Die Werte hängen also nur von $n$ und $k$ ab. Wir
Konzentrieren uns auf das Produkt und schreiben es so:
\begin{align*}
  \prod^n_{\substack{j=0\\j\neq k}} \frac{s-j}{k-j}\text{d}s &= \frac{(s-0)(s-1)\dots(s-(k-1))(s-(k+1))\dots(s-(n-1))(s-n)}
  {(k-0)(k-1)\dots(k-(k-1))(k-(k+1))\dots(k-(n-1))(k-n)}\\
  \intertext{Jetzt tauschen wir $k$ mit $n-k$}
  \prod^n_{\substack{j=0\\j\neq n-k}} \frac{s-j}{n-k-j}\text{d}s &= \frac{(s-0)(s-1)\dots(s-(n-k-1))(s-(n-k+1))\dots(s-(n-1))(s-n)}
  {(n-k-0)(n-k-1)\dots(n-k-(n-k-1))(n-k-(n-k+1))\dots(n-k-n)}
\end{align*}

Man sieht, dass die Brüche sich Spiegeln. $(k-0)$ ist äquivalent zu
$(n-k-n)=(-k)$ und die andere wirken ähnlich. Zwar ist die Zahl
negativ, aber das gleich sich aus.

\section*{3}

Wir definieren $N_{n+1}(x) = (x - x_0)\dots(x-x_n)$ und $R$ als
überbleibendes Polynom vom Grad $n$. Wir können das Polynom $p \in
P_{n+1}$ so definieren

\[
p = a_{n+1} \cdot N_{n+1} + R
\]

Der Fehler berechnen wir mit

\begin{align*}
  E_n(p) &= |I(p) - I_n(p)|\\
  \intertext{Aus der Exaktheit von Grad $n$ folgt}
  &= a_{n+1}\cdot\left[I\left(N_{n+1}\right) -
    I_n\left(N_{n+1}\right)\right]
\end{align*}

$N_{n+1}$ ist punktsymmetrisch zu $\frac{a+b}{2}$, und daher ist
$I(N_{n+1}) = 0$.\\

$I_n(N_{n+1})$ ist $(x_0 - x_n)\sum^n_{k=1} \lambda_k N_{n+1}(x_n)$
und wird auch Null da die Summe mit den Nullstellen des
Newton-Polynoms arbeitet.

\end{document}
