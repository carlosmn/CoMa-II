\documentclass[ngerman,a4paper]{scrartcl}
\usepackage{relsize}
\usepackage{fullpage}
\usepackage[german]{babel}
\usepackage{graphicx}
\usepackage{cancel}

%Compiler
\usepackage{ifxetex}
\usepackage{ifluatex}
\ifxetex
  \usepackage{fontspec,xunicode}
  \catcode`\ß=13
  \defß{\ss}
\else\ifluatex
  \usepackage{fontspec,xunicode}
\else
  \usepackage[utf8]{inputenc}
\fi\fi
% /Compiler

\usepackage{amsmath, amssymb, amsfonts, amsthm}
\usepackage{listings}
\lstset{frame=single}

\newcommand{\norm}[1]{\left|\!\left|#1\right|\! \right|}
\newcommand{\R}{\ensuremath{\mathbb{R}}}
\newcommand{\N}{\ensuremath{\mathbb{N}}}

\begin{document}
{\sffamily
  \hfill
  CoMa-II SS 2013\hfill
  FU Berlin\\[8pt]
  \noindent {\Huge Übung 6}\hfill Carlos Martín Nieto, Tran Tu\hrule \bigskip
}

\section*{1}

\subsection*{a)}

\paragraph{(1)}

Wir leiten $y$ ab

\begin{align*}
  F(t,y) &= \dot{y} = \left(\frac{1}{1+t^2}\right)' =   \left((1+t^2)^{-1}\right)' = -(1+t^2)^{-2}(2t)\\
&= -2t \cdot \frac{1}{(1+t^2)^2} = -2t \underbrace{ \left(\frac{1}{1+t^2} \right)^2}_{y(t)^2}
\intertext{$y(t)$ ist genau was das $y$ Parameter ist}
F(t,y) &= -2ty^2
\end{align*}

was den Wert in der Aufgabenstellung entspricht.

\paragraph{(2)}

Analog

\begin{align*}
  F(t,y) &= \dot{y} = (e^{-2t}+1)' = -2e^{-2t}
\intertext{Wir addieren Null}
&= -2e^{-2t} - 2 + 2 = -2(\underbrace{e^{-2t} + 1}_{y(t)} - 1) = -2(y -1) = -2y + 2\\
F(t,y) &= 2-2y
\end{align*}

was den Wert in der Aufgabenstellung entspricht.

\section*{2}

\paragraph{a)}

Sei $x(t)$ die Anzahl der Bakterien, und $p \Delta t$ die Wahrscheinlichkeit, dass eine Bakterie sich teilt. Mit $t > 0$. Man berechne die Anzahl der Bakterien so

\begin{align*}
  x(t + \Delta t) &= x(t) + p \Delta t \cdot x(t)\\
  px(t) &= \frac{x(t + \Delta t) - x(t)}{\Delta t}
  \intertext{Und mit $\Delta t \to 0$}
  px(t) &= x'(t)
\end{align*}

Somit wird unsere Differezialgleichung

\begin{align*}
  x'(t) &= px(t) \quad t > 0\\
  x(0) &= x_0
\end{align*}

\paragraph{b)}

Wir fügen $\Delta x_{\text{kon}}$ zu unsere Formen hinzu.


\begin{align*}
  x(t + \Delta t) &= x(t) + p \Delta t \cdot x(t) - kx(t)^2 \Delta t\\
  &= x(t) + \Delta t\left( px(t) - kx(t)^2 \right)\\
  px(t) - kx(t)^2 &= \frac{x(t + \Delta t) - x(t)}{\Delta t}
  \intertext{Und mit $\Delta t \to 0$}
  px(t) - kx(t)^2 &= x'(t)
\end{align*}

Somit wird unsere Differezialgleichung

\begin{align*}
  x'(t) &= px(t) - kx(t)^2\\
  x(0) &= x_0
\end{align*}

\paragraph{c)}

Die Lösung von \textbf{a)} folgt das Schema aus dem Beispiel im Skript.

Eine Losung davon ist $x(t) = x_0 e^{pt}$. Wir zeigen, dass die Lösung gültig ist

\begin{align*}
  \dot{x} &= (x_0 e^{pt})' = p \underbrace{x_0 e^{pt}}_{x(t)} = px(t)
\end{align*}

Was das $x'(t)$ der Differenzialgleichung entspricht.

\end{document}
