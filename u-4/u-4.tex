\documentclass[ngerman,a4paper]{scrartcl}
\usepackage{relsize}
\usepackage{fullpage}
\usepackage[german]{babel}
\usepackage{graphicx}

%Compiler
\usepackage{ifxetex}
\usepackage{ifluatex}
\ifxetex
  \usepackage{fontspec,xunicode}
  \catcode`\ß=13
  \defß{\ss}
\else\ifluatex
  \usepackage{fontspec,xunicode}
\else
  \usepackage[utf8]{inputenc}
\fi\fi
% /Compiler

\usepackage{amsmath, amssymb, amsfonts, amsthm}
\usepackage{listings}
\lstset{frame=single}

\newcommand{\norm}[1]{\left|\!\left|#1\right|\! \right|}
\newcommand{\R}{\ensuremath{\mathbb{R}}}
\newcommand{\N}{\ensuremath{\mathbb{N}}}

\begin{document}
{\sffamily
  \hfill
  Computerorientierete Mathematik SS 2013\hfill
  FU Berlin\\[8pt]
  \noindent {\Huge Übung 4}\hfill Carlos Martín Nieto, Tran Tu\hrule \bigskip
}

\section*{1}

Die Fehlerabschätzung wird im Skript so beschrieben

\[
\norm{f - p_n}_\infty \leq \frac{1}{(n+1)!} \norm{f^{(n+1)}}_\infty \norm{w}_\infty
\]

Unser $n=2$ und $\norm{w}_\infty$ ist auch aus dem Skript zu lesen:

\[
\norm{w}_\infty = \max_{x\in [a,b]} \left|(x-x_0) \dots (x-x_n)\right|
\]

Dies kann man leicht mit Matlab berechnen und kommt auf
$\norm{w}_\infty = 0,3849$. $f^{(3)}$ löst man und kommt auf

\[
f^{(3)} = \frac{1}{2^3} \exp(x/2)
\]

was man auch an Matlab übergibt, um auf $\norm{f^{(3)}} = 0,20609$ zu
kommen. Fügt man alles zusammen und wir kommen auf

\[
\norm{f - p_n}_\infty \leq \frac{0,20609}{3!} \cdot 0,3849 = \boxed{0,013221}
\]

\begin{lstlisting}[language=octave]
max(abs(arrayfun(@(x) (x - xs(1)) * (x - xs(2)) * (x - xs(3)),
    linspace(-1, 1, 1000))))
      -> 0,38490
max(arrayfun(@(x) (exp(x/2)), linspace(-1, 1, 1000)))/8
      -> 0,20609
\end{lstlisting}

\section*{2}

\paragraph{a)}

Die Behauptung ist, dass 

\[
\int^b_a f(x)\, dx \approx \int^b_a p_1(x)\, dx
\]

der Trapezregel entspricht. Wir formen beide Seiten um, um dies zu
Testen. Zuerst wenden wir die Trapezregel an.

\begin{align*}
  \int^a_b f(x)\, dx &= \frac{a-b}{2n} \big[ f(x_0) + f(x_1) \big] & n
  = 1,\, x_0 = a,\, x_1 = b\\
  &= \frac{a-b}{2} \big[f(a) + f(b)\big] \quad (*)
\end{align*}

Und von der anderen Seite schreiben wir es auch um

\begin{align*}
  \int^b_a p_1(x)\, dx &= (b-a) \sum^n_{k=0} f(x_k) \lambda_k & n = 1\\
  &= (b-a)\big[f(x_0)\lambda_0 + f(x_1)\lambda_1\big] &  x_0 = a,\, x_1 = b,\, \lambda_{0,1}
  = \frac{1}{2}\\
  &= (b-a)\big[f(a)\frac{1}{2} + f(b)\frac{1}{2}\big]\\
  &= \frac{a-b}{2} \big[f(a) + f(b)\big] \quad (**)
\end{align*}

Offensichtlich sind $(*)$ und $(**)$ gleich, also stimmt die Behauptung.

\paragraph{b)}

Die Behauptung ist, dass 

\[
\int^b_a f(x)\, dx \approx \int^b_a p_2(x)\, dx
\]

der Simpsonregel entspricht. Wir formen beide Seiten um, um dies zu
Testen. Zuerst wenden wir die Simpsonregel an.

\begin{align*}
  \int^a_b f(x)\, dx &= \frac{a-b}{6} \left[ f(a) +
    4f\left(\frac{a+b}{2}\right) + f(b)\right] \quad (*)
\end{align*}

Jetzt schreiben wir die Rechte Seite um

\begin{align*}
  \int^b_a p_2(x)\, dx &= (b-a) \sum^n_{k=0} f(x_k) \lambda_k & n = 2\\
  &= (b-a)\left[f(x_0)\lambda_0 + f\left(x_1\right)\lambda_1 +
    f(x_2)\lambda_2\right]\\
  &= (b-a)\left[f(a)\frac{1}{6} +
    f\left(\frac{1}{2}(a+b)\right)\frac{4}{6} +
    f(b)\frac{1}{6}\right]\\
  &= \frac{b-a}{6} \left[f(a) + f\left(\frac{a+b}{2}\right)\cdot 4 +
    f(b) \right] \quad (**)
\end{align*}

Auch hier sind $(*)$ und $(**)$ offensichtlich gleich, also stimmt
auch diese Behauptung.

\section*{3}

Aus der Vorlesung wissen wir, dass die Absolute kondition ist

\[
|I(f) - I(\tilde{f})| \leq \norm{f-\tilde{f}}_\infty (b-a)
\]

So ist die relative Kondition

\[
\frac{\norm{f - \tilde{f}} (b-a)}{\norm{f} (b-a)}
\]

\end{document}
