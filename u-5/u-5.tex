\documentclass[ngerman,a4paper]{scrartcl}
\usepackage{relsize}
\usepackage{fullpage}
\usepackage[german]{babel}
\usepackage{graphicx}

%Compiler
\usepackage{ifxetex}
\usepackage{ifluatex}
\ifxetex
  \usepackage{fontspec,xunicode}
  \catcode`\ß=13
  \defß{\ss}
\else\ifluatex
  \usepackage{fontspec,xunicode}
\else
  \usepackage[utf8]{inputenc}
\fi\fi
% /Compiler

\usepackage{amsmath, amssymb, amsfonts, amsthm}
\usepackage{listings}
\lstset{frame=single}

\newcommand{\norm}[1]{\left|\!\left|#1\right|\! \right|}
\newcommand{\R}{\ensuremath{\mathbb{R}}}
\newcommand{\N}{\ensuremath{\mathbb{N}}}

\begin{document}
{\sffamily
  \hfill
  Computerorientierete Mathematik SS 2013\hfill
  FU Berlin\\[8pt]
  \noindent {\Huge Übung 5}\hfill Carlos Martín Nieto, Tran Tu\hrule \bigskip
}

\section*{1}

Aus dem Skript erfahren wir die Fehlerabschätzung der summierten Trapetzregel, und zwar ist sie

\[
|I(f) - S^{(n)}_n(f)| \leq \frac{h^2}{12}(b-a)\norm{f''}_\infty
\]

Wir wollen, dass der Fehler unter $10^{-4}$ bleibt. $a$ und $b$ stehen fest, sowie $f$. Die freie Variable ist dann $h$. Wir rechnen mit diesen Werten und berechnen $h$.

\[
  10^{-4} = \frac{h^2}{12} (a-b) \norm{f''}_\infty
\]

Zuerst finden wir heraus, was $\norm{f''}_\infty$ ist


\begin{align*}
  f'' &= \sin(\pi x)'' = \pi \cos(\pi x)' = \pi \cdot \pi \cdot       [-\sin(\pi x)] = -\pi^2 \sin(\pi x)\\
\norm{f''}_\infty &= \norm{-\pi^2 \sin(\pi x)}_\infty
\end{align*}

Und da Sinus $[-1, 1]$ als Ausgabe hat, ist $\norm{f''}_\infty = \pi^2$. Setzen wir das in der Formel

\begin{align*}
  10^{-4} &= \frac{h^2}{12} (a-b) \norm{f''}_\infty\\
  10^{-4} &= \frac{h^2}{12} \underbrace{(a-b)}_{(1 - 0)} \pi^2 = \frac{\pi^2 h^2}{12}\\
\intertext{Da wir $h$ berechnen wollen}
h^2 &= \frac{12\cdot 10^{-4}}{\pi^2}\\
h &= \sqrt{\frac{12\cdot 10^{-4}}{\pi^2}} = \frac{\sqrt{12\cdot 10^{-4}}}{\pi}\\
h &= \frac{0.034641}{\pi} = 0.011027
\end{align*}

Diese ist die maximale Größe eines Teilintervals, um die gewünschte Genauigkeit zu erreichen. $h$ ist aber eigentlich ein Stellvertreter für $n$, die Anzahl der Teilintervale, was eine natürliche Zahl ist

\[
n = \left\lceil \frac{1}{h} \right\rceil = \left\lceil \frac{1}{0.011027} \right\rceil = \left\lceil 90.686 \right\rceil = 91
\]

Wir brauchen also mindestens 91 Teilintervale, um die gewünschte Genauigkeit zu erreichen, was $h$ zu 0.010989 bringt.

\end{document}
