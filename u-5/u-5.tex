\documentclass[ngerman,a4paper]{scrartcl}
\usepackage{relsize}
\usepackage{fullpage}
\usepackage[german]{babel}
\usepackage{graphicx}
\usepackage{cancel}

%Compiler
\usepackage{ifxetex}
\usepackage{ifluatex}
\ifxetex
  \usepackage{fontspec,xunicode}
  \catcode`\ß=13
  \defß{\ss}
\else\ifluatex
  \usepackage{fontspec,xunicode}
\else
  \usepackage[utf8]{inputenc}
\fi\fi
% /Compiler

\usepackage{amsmath, amssymb, amsfonts, amsthm}
\usepackage{listings}
\lstset{frame=single}

\newcommand{\norm}[1]{\left|\!\left|#1\right|\! \right|}
\newcommand{\R}{\ensuremath{\mathbb{R}}}
\newcommand{\N}{\ensuremath{\mathbb{N}}}

\begin{document}
{\sffamily
  \hfill
  CoMa-II SS 2013\hfill
  FU Berlin\\[8pt]
  \noindent {\Huge Übung 5}\hfill Carlos Martín Nieto, Tran Tu\hrule \bigskip
}

\section*{1}

\paragraph{a)}

Aus dem Skript erfahren wir die Fehlerabschätzung der summierten Trapetzregel, und zwar ist sie

\[
|I(f) - S^{(1)}_n(f)| \leq \frac{h^2}{12}(b-a)\norm{f''}_\infty
\]

Wir wollen, dass der Fehler unter $10^{-4}$ bleibt. $a$ und $b$ stehen fest, sowie $f$. Die freie Variable ist dann $h$. Wir rechnen mit diesen Werten und berechnen $h$.

\[
  10^{-4} = \frac{h^2}{12} (b-a) \norm{f''}_\infty
\]

Zuerst finden wir heraus, was $\norm{f''}_\infty$ ist


\begin{align*}
  f'' &= \sin(\pi x)'' = \pi \cos(\pi x)' = \pi \cdot \pi \cdot       [-\sin(\pi x)] = -\pi^2 \sin(\pi x)\\
\norm{f''}_\infty &= \norm{-\pi^2 \sin(\pi x)}_\infty
\end{align*}

Und da Sinus $[-1, 1]$ als Ausgabe hat, ist $\norm{f''}_\infty = \pi^2$. Setzen wir das in der Formel

\begin{align*}
  10^{-4} &= \frac{h^2}{12} (b-a) \norm{f''}_\infty\\
  10^{-4} &= \frac{h^2}{12} \underbrace{(b-a)}_{(1 - 0)} \pi^2 = \frac{\pi^2 h^2}{12}\\
\intertext{Da wir $h$ berechnen wollen}
h^2 &= \frac{12\cdot 10^{-4}}{\pi^2}\\
h &= \sqrt{\frac{12\cdot 10^{-4}}{\pi^2}} = \frac{\sqrt{12\cdot 10^{-4}}}{\pi}\\
h &= \frac{0.034641}{\pi} = 0.011027
\end{align*}

Diese ist die maximale Größe eines Teilintervals, um die gewünschte Genauigkeit zu erreichen. $h$ ist aber eigentlich ein Stellvertreter für $n$, die Anzahl der Teilintervale, was eine natürliche Zahl ist

\[
n = \left\lceil \frac{1}{h} \right\rceil = \left\lceil \frac{1}{0.011027} \right\rceil = \left\lceil 90.686 \right\rceil = 91
\]

Wir brauchen also mindestens 91 Teilintervale, um die gewünschte Genauigkeit zu erreichen, was $h$ zu 0.010989 bringt.

\paragraph{b)}

Wir haben eine Quadraturformel der Form

\[
\sum^2_{k=0} \lambda_k f(x_k)
\]

wobei die Gewichte die Form

\[
\lambda_k = \frac{1}{b-a} \int\limits^b_a L_k(x) dx
\]

annehmen. Also berechnen wir die Gewichte. $\lambda_0$ wird mit $f(x_0) = f(0) = 0$ multipliziert, also müssen wir nicht berechnen. Wir fangen mit den Lagrangepolynome an.

\begin{align*}
  L_k(x) &= \prod^n_{\substack{i=0\\i\neq k}} \frac{x - x_i}{x_k-x_i}         \quad k = 0,\dots,n\\
L_1(x) &= \frac{x - 0}{\frac{1}{4} - 0} \cdot \frac{x - 1}{\frac{1}{4} -1} = \frac{x^2 - x}{\frac{-3}{16}}\\
L_2(x) &= \frac{x-0}{1 - 0}\cdot \frac{x - \frac{1}{4}}{1 - \frac{1}{4}} = x \cdot \frac{x - \frac{1}{4}}{\frac{3}{4}} = \frac{x^2 - \frac{1}{4}x}{\frac{3}{4}}
\end{align*}

Und jetzt die Gewichte. $\frac{1}{b-a} = \frac{1}{1-0} = 1$ also lassen wir es weg von der Berechnungen.

\begin{align*}
  \lambda_k &= \int_0^1 L_k(x) dx\\
  \lambda_1 &= \left.\frac{\frac{1}{3}x^3 - \frac{1}{2} x^2}{\frac{-3}{16}}\right|_0^1 = \frac{\frac{1}{3} - \frac{1}{2}}{\frac{-3}{16}} = \frac{8}{9}\\
\lambda_2 &= \left. \frac{\frac{1}{3}x^3 - \frac{1}{8}x^2}{\frac{3}{4}} \right|_0^1 = \frac{\frac{1}{3}-\frac{1}{8}}{\frac{3}{4}} = \frac{5}{18}
\end{align*}

Jetzt fehlt nur einsetzen.

\begin{align*}
  \int\limits_0^1 f(x) &\approx \sum^2_{k=0} \lambda_k f(x_k) = \cancelto{0}{\lambda_0 f(x_0)} + \lambda_1 f(x_1) + \lambda_2 f(x_2)\\
&= \underbrace{\frac{8}{9}}_{\frac{16}{18}} f\left(\frac{1}{4}\right) + \frac{5}{18} f(1) = \frac{16}{18} f\left(\frac{1}{4}\right) + \frac{5}{18} f(1)\\
&= \frac{1}{18}\left[ 16 f\left(\frac{1}{4}\right) + 5 f(1) \right]
\end{align*}

\end{document}
